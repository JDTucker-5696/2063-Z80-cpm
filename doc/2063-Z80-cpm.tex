\documentclass[10pt,letterpaper]{article}
\textwidth = 6.5in
\textheight = 9in
\hoffset=-.75in
\voffset=-.8in

\usepackage{ifthen}
\usepackage{stringstrings}  % so can count characters in a string
\usepackage{xstring}        % so can count characters in a string


\usepackage[pass]{geometry}
%\usepackage[hypertex]{hyperref}
\usepackage{hyperref}
\usepackage{lastpage}
\usepackage{fancyhdr}
\usepackage{sectsty}
\usepackage{amsmath}
\usepackage{scrextend}

\usepackage{listings}
\usepackage{xcolor}
\usepackage{graphicx}
\usepackage{epsfig}

%\graphicspath{ {./images/} }
\usepackage{tikz}
\usepackage{tikz-qtree}
\usepackage{tikz-timing}
	\usetikzlibrary{arrows.meta}
	%\usetikztiminglibrary[simple]{advnodes}
	\usetikztiminglibrary{advnodes}
	\usetikzlibrary{automata, positioning, arrows}
	\usetikzlibrary{decorations.pathreplacing,calligraphy}
\usepackage{enumitem}
\usepackage{placeins}





\sectionfont{\Large\sf\bfseries}
\subsectionfont{\large\sf\bfseries}

\pagestyle{fancy}
% supress normal headings and footters
\fancyhf{}
% remove the heading rule
\renewcommand{\headrulewidth}{0pt}

\lfoot{%{\sf\scriptsize Copyright \copyright\ 2021 John Winans.  All Rights Reserved}\\
{\scriptsize\FooterText}}
%\lfoot{\scriptsize\FooterText}

\rfoot{Page \thepage\ of \pageref*{LastPage}}

% Sub-footer that shows the VCS Header in the lfoot defined above
\ifdefined\GitFileName
    \newcommand{\FooterText}{\tt \GitFileName\\
\GitDescription}
\else
    \newcommand{\FooterText}{\emph{--UNKNOWN--}}
\fi


\setlength{\parindent}{0pt}
\setlength{\parskip}{.51em}


%%%%%%%%%%%%%%%%%%%%%%%%%%%%%%%%%%%%%%%%%%%%%%%%%%%%%%%%%%%%%%%%%%%%
\definecolor{c_lightblue}{HTML}{B0E0FF}
\definecolor{c_lightred}{HTML}{FFE0E0}
\definecolor{c_lightyellow}{HTML}{FFE060}
\definecolor{c_lightgreen}{HTML}{C0FFC0}
\definecolor{c_lightgray}{HTML}{C0C0C0}
%%%%%%%%%%%%%%%%%%%%%%%%%%%%%%%%%%%%%%%%%%%%%%%%%%%%%%%%%%%%%%%%%%%%

%%%%%%%%%%%%%%%%%%%%%%%%%%%%%%%%%%%%%%%%%%%%%%%%%%%%%%%%%%%%%%%%%%%%
%%%%%%%%%%%%%%%%%%%%%%%%%%%%%%%%%%%%%%%%%%%%%%%%%%%%%%%%%%%%%%%%%%%%
%%%%%%%%%%%%%%%%%%%%%%%%%%%%%%%%%%%%%%%%%%%%%%%%%%%%%%%%%%%%%%%%%%%%
%%%%%%%%%%%%%%%%%%%%%%%%%%%%%%%%%%%%%%%%%%%%%%%%%%%%%%%%%%%%%%%%%%%%
%%%%%%%%%%%%%%%%%%%%%%%%%%%%%%%%%%%%%%%%%%%%%%%%%%%%%%%%%%%%%%%%%%%%

\begin{document}
\thispagestyle{fancy}

\begin{center}
{\huge Z80-Retro! Memory Layout}
\end{center}
\vspace{.5in}


%%%%%%%%%%%%%%%%%%%%%%%%%%%%%%%%%%%%%%%%%%%%%%%%%%%%%%%%%%%%%%%%%%%%%%%%%%%%%
\begin{tikzpicture}[line width=.3mm]%[x=.4cm,y=.3cm]

	% draw backgrounds first so is behind everything else
	\fill [green] (1,12) rectangle (3,16);		% 16K load region
	\fill [yellow] (1,8) rectangle (3,13.25);	% TPA hi
	\fill [green] (2.6,12) rectangle (3,16);	% 16K load region (left overlay)
	\fill [yellow] (1,8) rectangle (1.4,14.6);	% TPA hi (left overlay)

	\fill [cyan] (9,0) rectangle (11,1);	% zero-page in bank 14
	\fill [yellow] (11,0) rectangle (13,4);	% TPA hi
	\fill [yellow] (9,1) rectangle (13,8);	% TPA lo

	% draw the colors into bank 15 as well to reinforce it is a copy
	\fill [green] (11,4) rectangle (13,8);		% 16K load region
	\fill [yellow] (11,0) rectangle (13,5.25);	% TPA hi
	\fill [green] (12.6,4) rectangle (13,8);	% 16K load region (right overlay)
	\fill [yellow] (11,0) rectangle (11.4,6.6);	% TPA hi (right overlay)


	\draw (1, 0) -- (1, 16);	% 64K left line
	\draw (3, 0) -- (3, 16);	% 64K right line


	\draw (1-.2,16) -- (3,16);	% top horiz
	\node [text width=4cm, align=right] at (-1.3, 16-.2) {\tt 0xFFFF};

	\draw (1-.2,14.625) -- (3,14.625);	% BIOS
	\node [text width=4cm, align=right] at (-1.3, 14.625+.2) {\tt 0xEA00};
	\node at (2,15.1+.2) {BIOS};

	\draw (1-.2,13.75) -- (3,13.75);	% BDOS
	\node [text width=4cm, align=right] at (-1.3, 13.75+.2) {\tt FBASE = 0xDC00};
	\node at (2,14+.2) {BDOS};

	\draw (1-.2,13.25) -- (3,13.25);	% CPP
	\node [text width=4cm, align=right] at (-1.3, 13.25+.2) {\tt CPM\_BASE = 0xD400};
	\node at (2,13.25+.2) {CCP};

	\draw (1-.2,12) -- (3,12);	% LOAD_BASE
	\node [text width=4cm, align=right] at (-1.3, 12+.2) {\tt LOAD\_BASE = 0xC000};

	\draw [decorate, decoration = {calligraphic brace}] (3.2,16) -- (3.2,12);
	\node [text width=2cm, align=left] at (4.3,14) {16KiB};

	\draw [decorate, decoration = {calligraphic brace}] (4.5,16) -- (4.5,8);
	\node [text width=4cm, align=left] at (6.65,12) {32KiB (Bank 15 Copy)};

	% Draw an arrow showing the copy of Bank 15
	%\draw[blue,->](12,8.1) to [out=90,in=0](8.6,12);
	%\draw[blue,->](13.6,4) to [out=80,in=0](8.6,12);
	\draw[ultra thick, blue,->](13.6,4) .. controls (16,6.4) and (16,13) .. (8.4,12);	


	% label the regions in bank 15 too
	\draw (11,6.625) -- (13,6.625);	% BIOS
	\node at (12,7.1+.2) {BIOS};
	\draw (11,5.75) -- (13,5.75);	% BDOS
	\node at (12,6+.2) {BDOS};
	\draw (11,5.25) -- (13,5.25);	% CPP
	\node at (12,5.25+.2) {CCP};
	\draw (11,4) -- (13,4);	% LOAD_BASE


	\draw (1-.2,1) -- (,1);	% TPA bottom
	\draw (9,1) -- (11,1);	% TPA bottom
	\node [text width=4cm, align=right] at (-1.3, 1+.2) {\tt TPA\_BASE = 0x0100};
	\draw [decorate, decoration = {calligraphic brace}] (-.45,1) -- (-.45,13.2);
	\node [text width=.9cm, align=right] at (-1.1, 7.1) {\small\it TPA};

	\draw [decorate, decoration = {calligraphic brace}] (-.45,13.3) -- (-.45,16);
	\node [text width=.9cm, align=right] at (-1.1, 14.6) {\small\it O/S};


	\draw (1-.2,.5) -- (1,.5);	% default DMA buffer
	\draw (9,.5) -- (11,.5);	% default DMA buffer
	\node [text width=4cm, align=right] at (-1.3, .5+.2) {\tt 0x0080};
	\node at (10,.5+.2) {I/O BUF};

	\draw (1-.2,0) -- (3,0);	% bottom horiz
	\node [text width=4cm, align=right] at (-1.3, 0+.2) {\tt BOOT = 0x0000};


	\node [text width=4cm, align=right] at (-1.3, 8+.2) {\tt 0x8000};

	% le'other 15 memory banks
	\foreach \x in {1,...,5}%
		\draw (3+2*\x, 0) -- (3+2*\x, 8);	% 32K bank right line

	\node at (2, 3) {Bank 0};
	\node at (4, 3) {Bank 1};
	\node at (6, 3) {Bank 2};
	\node at (8, 3) {\ldots};
	\node at (10, 3) {Bank 14};
		\node at (10, 2) {TPA lo};
	\node at (12, 3) {Bank 15};
		\node at (12, 2) {TPA hi};	% in right-side bank 15
		\node at (2, 10) {TPA hi};	% in top-copy of bank 15

	\draw (3,0) -- (7.9,0);		% 32K bottom horizontal
	\draw (8.1,0) -- (3+2*5,0);	% 32K bottom horizontal
	\draw (7.9,-.1) -- (7.9,.1); % vert-left marker
	\draw (8.1,-.1) -- (8.1,.1); % vert-right marker

	\draw[densely dotted] (1,8) -- (7.9,8);		% 32K top horizontal
	\draw[densely dotted] (8.1,8) -- (3+2*5,8);	% 32K top horizontal
	\draw[densely dotted] (7.9,7.9) -- (7.9,8.1); % vert-left marker
	\draw[densely dotted] (8.1,7.9) -- (8.1,8.1); % vert-right marker

	\draw [decorate, decoration = {calligraphic brace, mirror}] (13.4,0) -- (13.4,8);
	\node at (14.6,4) {\parbox{2cm}{32KiB}};


	\draw [decorate, decoration = {calligraphic brace}] (9-.05,-.3) -- (1,-.3);		% bottom cache
	\node [text width=4cm, align=center] at (5, -.7) {I/O Cache};

	\draw [decorate, decoration = {calligraphic brace}] (13,-.3) -- (9+.05,-.3);	% bottom main memory
	\node [text width=4cm, align=center] at (11, -.7) {Main Memory};

\end{tikzpicture}


\clearpage

\begin{center}
{\huge Z80-Retro! SD Card Layout}
\end{center}
\vspace{.5in}


%%%%%%%%%%%%%%%%%%%%%%%%%%%%%%%%%%%%%%%%%%%%%%%%%%%%%%%%%%%%%%%%%%%%%%%%%%%%%
\begin{tikzpicture}[line width=.3mm]%[x=.4cm,y=.3cm]

	% draw backgrounds first so is behind everything else

	\node at (1.5, 17-.2) {\bfseries 8MiB CP/M Disk Image};

	\fill [green] (1,0) rectangle (3,.5);	% the O/S (the bottom 32 blocks)
	\fill [yellow] (1,.5) rectangle (3,1);	% the blocks for the directory (the next 32 blocks)
	\fill [orange] (1,1) rectangle (3,16);	% the data blocks 

	\draw (1, 0) -- (1, 16);	% 64K left line
	\draw (3, 0) -- (3, 16);	% 64K right line

	\node at (-.2, 16.5-.2) {SD Block};

	\draw (1-.2,16) -- (3,16);	% top horiz
	\node at (-.2, 16-.2) {\tt 0x3fff};

	\draw (1-.2,1) -- (3,1);	% Data blocks
	\node at (-.2, 1+.2) {\tt 0x0040};
	\node at (2,8+.2) {DATA};

	\draw (1-.2,.5) -- (3,.5);	% default DMA buffer
	\node at (-.2, .5+.2) {\tt 0x0020};
	\node at (2,.5+.2) {DIR};

	\draw (1-.2,0) -- (3,0);	% bottom horiz
	\node at (-.2, 0+.2) {\tt 0x0000};
	\node at (2,0+.2) {O/S};


	\begin{scope}[xshift=9cm]
		\fill [cyan] (1,0) rectangle (3,.5);	% the MBR
		\fill [yellow] (1,1) rectangle (3,2);	% Partition 1

		\node at (1.5, 17-.2) {\bfseries 16GiB SD Card};

		\draw (1, 0) -- (1, 16);	% 64K left line
		\draw (3, 0) -- (3, 16);	% 64K right line

		\node at (-.2, 16.5-.2) {SD Block};

		\draw (1-.2,16) -- (3,16);	% top horiz
		\node at (-.2, 16-.2) {\tt 0x01ffffff};


		\draw (1-.2,2) -- (3,2);
		\node at (-.2, 2+.2) {\tt 0x00040800};

		\draw[dotted] (1-.2,1.1) coordinate (p1top) -- (3,1.1);

		\draw (1-.2,1) coordinate (p1bot) -- (3,1);
		\node at (-.2, 1+.2) {\tt 0x00000800};
		\node at (2,1.5) {P1};

		\draw [decorate, decoration = {calligraphic brace}] (3.1,2) -- (3.1,1);
		\node at (4.3,1.5) {\parbox{2cm}{128MiB}};

		\draw (1-.2,.5) -- (3,.5);
		\node at (-.2, .5+.2) {\tt 0x00000001};

		\draw [decorate, decoration = {calligraphic brace}] (3.1,.5) -- (3.1,0);
		\node at (4.3,.25) {\parbox{2cm}{512B}};

		\draw (1-.2,0) -- (3,0);	% bottom horiz
		\node at (-.2, 0+.2) {\tt 0x00000000};
		\node at (2,0+.2) {MBR};
	\end{scope}

	% expansion lines
	\draw[densely dotted] (3, 16)to[out=0,in=180](p1top);
	\draw[densely dotted] (3, 0)to[out=0,in=180](p1bot);

\end{tikzpicture}





%%%%%%%%%%%%%%%%%%%%%%%%%%%%%%%%%%%%%%%%%%%%%%%%%%%%%%%%%%%%%%%%%%%%%%%%%%%%%






\def\SignBoxCornerRadius{.75mm}

%%%%%%%%%%%%%%%%%%%%%%%%%%%%%%%%%%%%%%%%%%%%%%%%%%%%%%%%%%%%%
\newcommand\BeginTikzPicture{
    \begin{tikzpicture}[x=.4cm,y=.3cm]
}

%%%%%%%%%%%%%%%%%%%%%%%%%%%%%%%%%%%%%%%%%%%%%%%%%%%%%%%%%%%%%
\newcommand\EndTikzPicture{
    \end{tikzpicture}
}

%%%%%%%%%%%%%%%%%%%%%%%%%%%%%%%%%%%%%%%%%%%%%%%%%%%%%%%%%%%%%
% Print the characters within a string evenly spaced at integral node positions
%
% #1 The number of characters in the string
% #2 The string to print
\newcommand\DrawBitstring[2]{
\foreach \x in {1,...,#1}%
	\draw(\x,0) node{\substring{#2}{\x}{\x}};%
%	\draw(\x,.5) node[text width = 10, text height = 1]{\substring{#2}{\x}{\x}};%	Improve vertical text alignment
}

%%%%%%%%%%%%%%%%%%%%%%%%%%%%%%%%%%%%%%%%%%%%%%%%%%%%%%%%%%%%%
% #1 The total size
% #2 The string to print
% #3 The value to use when extending to left
\newcommand\DrawLeftExtendedBitstring[3]{
	\StrLen{#2}[\numchars]

	\pgfmathsetmacro\leftpadd{int(#1-\numchars)}
	\foreach \x in {1,...,\leftpadd}
    	\draw(\x,0) node{#3};

	\pgfmathsetmacro\leftpadd{int(\leftpadd+1)}
	\foreach \x in {\leftpadd,...,#1}
		\pgfmathsetmacro\ix{int(\x-\leftpadd+1)}
		\draw(\x,0) node{\substring{#2}{\ix}{\ix}};
}

%%%%%%%%%%%%%%%%%%%%%%%%%%%%%%%%%%%%%%%%%%%%%%%%%%%%%%%%%%%%%
% If the string is shorter than expected, extend with #5 to the right.
%
% #1 The total size
% #2 Num chars to extend on the right
% #3 The string to print
% #4 The value to use when extending to left
% #5 The value to use when extending to right
\newcommand\DrawDoubleExtendedBitstring[5]{
	\StrLen{#3}[\numchars]

	\pgfmathsetmacro\leftpadd{int(#1-#2-\numchars)}
	\ifthenelse{1 > \leftpadd}
	{}
	{
		\foreach \x in {1,...,\leftpadd}
    		\draw(\x,0) node{#4};
	}

	\pgfmathsetmacro\leftpadd{int(\leftpadd+1)}
	\pgfmathsetmacro\rightpadd{int(\leftpadd+\numchars)}
	\foreach \x in {\leftpadd,...,\rightpadd}
		\pgfmathsetmacro\ix{int(\x-\leftpadd+1)}
		\draw(\x,0) node{\substring{#3}{\ix}{\ix}};


	%\pgfmathsetmacro\rightpadd{int(\rightpadd+1)}
	\ifthenelse{\rightpadd > #1}	
	{}
	{
		\foreach \x in {\rightpadd,...,#1}
			\draw(\x,0) node{#5};
	}
}




%%%%%%%%%%%%%%%%%%%%%%%%%%%%%%%%%%%%%%%%%%%%%%%%%%%%%%%%%%%%%
% Draw a box suitable to show the given number of bits in a 
% labeled box suitable for showing expanded binary numbers.
%
% #1 The number of characters to display
\newcommand\DrawBitBox[1]{
    \draw (.5,-.75) -- (#1+.5,-.75);		% box bottom
    \draw (.5,.75) -- (#1+.5,.75);			% box top
    \draw (.5,-.75) -- (.5, 1.5);			% left end
    \draw (#1+.5,-.75) -- (#1+.5, 1.5);		% right end
    \pgfmathsetmacro\result{int(#1-1)}		% calc high bit 
    \node at (1,1.2) {\tiny\result};		% high bit label
    \draw(#1,1.2) node{\tiny0};				% low bit label

    \pgfmathsetmacro\result{#1/2}
    \node at (\result,-1.2) {\tiny#1};		% size below the box

    \pgfmathsetmacro\result{#1/2}
    \draw[->] (\result+.6,-1.2) -- (#1+.5,-1.2);
    \draw[->] (\result-.6,-1.2) -- (.5,-1.2);
}


%%%%%%%%%%%%%%%%%%%%%%%%%%%%%%%%%%%%%%%%%%%%%%%%%%%%%%%%%%%%%
\newcommand\DrawBitBoxUnsigned[1]{
	\StrLen{#1}[\numchars]
	\DrawBitBox{\numchars}
	\DrawBitstring{\numchars}{#1}		% show the bits
}



%%%%%%%%%%%%%%%%%%%%%%%%%%%%%%%%%%%%%%%%%%%%%%%%%%%%%%%%%%%%%
\newcommand\DrawBitBoxUnsignedPicture[1]{
	\BeginTikzPicture
	\DrawBitBoxUnsigned{#1}
	\EndTikzPicture
}

%%%%%%%%%%%%%%%%%%%%%%%%%%%%%%%%%%%%%%%%%%%%%%%%%%%%%%%%%%%%%
\newcommand\DrawBitBoxSignedPicture[1]{
    \BeginTikzPicture
    \DrawBitBoxUnsigned{#1}
	% draw a box around the sign bit
	\draw {[rounded corners=\SignBoxCornerRadius] (1.35, -.6) -- (1.35, .6) -- (.65, .6) -- (.65, -.6) -- cycle};
    \EndTikzPicture
}


%%%%%%%%%%%%%%%%%%%%%%%%%%%%%%%%%%%%%%%%%%%%%%%%%%%%%%%%%%%%%
% #1 The total (extended) size
% #2 The value to use for left-side padding
% #3 The string to extend
\newcommand\DrawBitBoxLeftExtended[3]{
	\StrLen{#3}[\numchars]
	\pgfmathsetmacro\fill{int(#1-\numchars)}
	\begin{scope}[shift={(\fill,3.5)}]
    \DrawBitBoxUnsigned{#3}

   	% XXX IFF not zero-extending then draw a box around the sign bit
   	\draw {[rounded corners=\SignBoxCornerRadius] (1.35, -.6) -- (1.35, .6) -- (.65, .6) -- (.65, -.6) -- cycle};
	\end{scope}

	\DrawBitBox{#1}
	\DrawDoubleExtendedBitstring{#1}{0}{#3}{#2}{x}
	
   	% XXX IFF not zero-extending then draw a box around the sign bit
   	\draw {[rounded corners=\SignBoxCornerRadius] (\fill+1.35, -.6) -- (\fill+1.35, .6) -- (\fill+.65, .6) -- (\fill+.65, -.6) -- cycle};
    % draw a box around the extended sign bits
    \draw (.65, -.6) -- (.65, .6) -- (\fill+.35, .6) -- (\fill+.35, -.6) -- cycle;


}

%%%%%%%%%%%%%%%%%%%%%%%%%%%%%%%%%%%%%%%%%%%%%%%%%%%%%%%%%%%%%
\newcommand\DrawBitBoxSignExtendedPicture[2]{
	\BeginTikzPicture
	\DrawBitBoxLeftExtended{#1}{\substring{#2}{1}{1}}{#2}
    \EndTikzPicture
}

%%%%%%%%%%%%%%%%%%%%%%%%%%%%%%%%%%%%%%%%%%%%%%%%%%%%%%%%%%%%%
\newcommand\DrawBitBoxZeroExtendedPicture[2]{
	\BeginTikzPicture
	\DrawBitBoxLeftExtended{#1}{0}{#2}
    \EndTikzPicture
}


%%%%%%%%%%%%%%%%%%%%%%%%%%%%%%%%%%%%%%%%%%%%%%%%%%%%%%%%%%%%%
% #1 Total bit length
% #2 The string to print
% #3 Right-side padding length
\newcommand\DrawBitBoxSignLeftZeroRightExtendedPicture[3]{
	\BeginTikzPicture

	\StrLen{#2}[\numchars]
	\pgfmathsetmacro\fill{int(#1-\numchars-#3)}
	\begin{scope}[shift={(\fill,3.5)}]
    \DrawBitBoxUnsigned{#2}
    % draw a box around the sign bit
    %\draw (1.35, -.6) -- (1.35, .6) -- (.65, .6) -- (.65, -.6) -- cycle;
    \draw {[rounded corners=\SignBoxCornerRadius] (1.35, -.6) -- (1.35, .6) -- (.65, .6) -- (.65, -.6) -- cycle};
	\end{scope}

	\DrawBitBox{#1}
	\DrawDoubleExtendedBitstring{#1}{#3}{#2}{\substring{#2}{1}{1}}{0}

	% Box the sign bit
    \draw {[rounded corners=\SignBoxCornerRadius] (\fill+1.35, -.6) -- (\fill+1.35, .6) -- (\fill+.65, .6) -- (\fill+.65, -.6) -- cycle};

	\ifthenelse{\fill > 0}
	{
    	% Box the left-extended sign bits
    	\draw (.65, -.6) -- (.65, .6) -- (\fill+.35, .6) -- (\fill+.35, -.6) -- cycle;
		% \fill[blue!40!white] (.65, -.6) rectangle (\fill-.25, 1.2);
	}
	{}
	\ifthenelse{#3 > 0}
	{
    	% Box the right-extended sign bits
		\pgfmathsetmacro\posn{int(\numchars+\fill)}
    	\draw (\posn+.65, -.6) -- (\posn+.65, .6) -- (\posn+#3+.35, .6) -- (\posn+#3+.35, -.6) -- cycle;
	}
	{}
	

    \EndTikzPicture
}


%%%%%%%%%%%%%%%%%%%%%%%%%%%%%%%%%%%%%%%%%%%%%%%%%%%%%%%%%%%%%%%%%%%%%%%%%%%%%%
%%%%%%%%%%%%%%%%%%%%%%%%%%%%%%%%%%%%%%%%%%%%%%%%%%%%%%%%%%%%%%%%%%%%%%%%%%%%%%
%%%%%%%%%%%%%%%%%%%%%%%%%%%%%%%%%%%%%%%%%%%%%%%%%%%%%%%%%%%%%%%%%%%%%%%%%%%%%%
%%%%%%%%%%%%%%%%%%%%%%%%%%%%%%%%%%%%%%%%%%%%%%%%%%%%%%%%%%%%%%%%%%%%%%%%%%%%%%


%%%%%%%%%%%%%%%%%%%%%%%%%%%%%%%%%%%%%%%%%%%%%%%%%%%%%%%%%%%%%%%%%%%%%%%%%%%%%%
% Print the characters within a string evenly spaced at integral node positions
%
% #1 The number of characters in the string
% #2 The string of characters to plot
% #3 Right-side label
\newcommand\DrawInsnBitstring[3]{
	\pgfmathsetmacro\num{int(#1-1)}
	\foreach \x in {1,2,...,#1}
    	\draw(\x+.25,-.3) node[text width = 10, text height = 1]{\substring{#2}{\x}{\x}};
	\draw(#1+1,0) node[right]{#3};
}

%%%%%%%%%%%%%%%%%%%%%%%%%%%%%%%%%%%%%%%%%%%%%%%%%%%%%%%%%%%%%
% #1 total characters/width
% #2 MSB position
% #3 LSB position
% #4 the segment label
\newcommand\DrawInsnBoxSeg[4]{
	\pgfmathsetmacro\leftpos{int(#1-#2)}
	\pgfmathsetmacro\rightpos{int(#1-#3)}

	\draw (\leftpos-.5,-.75) -- (\rightpos+.5,-.75);	% box bottom
	\draw (\leftpos-.5,1.75) -- (\rightpos+.5,1.75);	% box top
	\draw (\leftpos-.5,-.75) -- (\leftpos-.5, 2.5);		% left end
	\draw (\rightpos+.5,-.75) -- (\rightpos+.5, 2.5);	% right end
	\node at (\leftpos,2.2) {\tiny#2};
	\draw(\rightpos,2.2) node{\tiny#3};

	\pgfmathsetmacro\posn{#1-#2+(#2-#3)/2}
	\node at (\posn,1.2) {\small#4};			% the field label

	\begin{scope}[shift={(0,-.7)}]\InsnBoxFieldWidthArrow{#1}{#2}{#3}\end{scope}
}

%%%%%%%%%%%%%%%%%%%%%%%%%%%%%%%%%%%%%%%%%%%%%%%%%%%%%%%%%%%%%
%%%%%%%%%%%%%%%%%%%%%%%%%%%%%%%%%%%%%%%%%%%%%%%%%%%%%%%%%%%%%
\newcommand\InsnStatement[1]{
%	\textbf{\large #1}\\
%	\textbf{#1}\\
	{\large #1}
}

%%%%%%%%%%%%%%%%%%%%%%%%%%%%%%%%%%%%%%%%%%%%%%%%

\newcommand\InsnBoxFieldWidthArrowVskip{.5}
\newcommand\InsnBoxFieldWidthArrowHskip{.05}

% #1 number of bits-wide
% #2 MSB position
% #3 LSB position
\newcommand\InsnBoxFieldWidthArrow[3]{
	\pgfmathsetmacro\leftpos{int(#1-#2-1)}		% Calculate the left end position
	\pgfmathsetmacro\wid{int(#2-#3+1)}			% calculate the width
	\begin{scope}[shift={(\leftpos,-\InsnBoxFieldWidthArrowVskip)}]	% Move to left end of arrow & below origin
    	\pgfmathsetmacro\result{\wid*.5+.5}		% the center position
    	\node at (\result,0) {\tiny\wid};		% draw the size number below the box

		\ifthenelse{\wid > 9}					% make 1-9 narrower than 10-99
		{ \pgfmathsetmacro\Inset{0.4} }
		{ 
			\ifthenelse{\wid > 1} 				% make 1 narrower than 2-9
			{ \pgfmathsetmacro\Inset{0.25} }
			{ \pgfmathsetmacro\Inset{0.15} }
		}

		% arrowsInsnBoxFieldWidthArrowHskip
    	\draw[->] (\result+\Inset,0) -- (\wid+.5-\InsnBoxFieldWidthArrowHskip,0);	% arrow to the right
    	\draw[->] (\result-\Inset,0) -- (.5+\InsnBoxFieldWidthArrowHskip,0);		% arrow to the left

		\pgfmathsetmacro\x{.5}
		\pgfmathsetmacro\y{\InsnBoxFieldWidthArrowVskip}
		% vertical bars at the ends of the arrows
    	\draw[-] (\x,\y) -- (\x,-\y*.5);	
		\pgfmathsetmacro\x{(\wid+.5}
    	\draw[-] (\x,\y) -- (\x,-\y*.5);	

	\end{scope}
}


%%%%%%%%%%%%%%%%%%%%%%%%%%%%%%%%%%%%%%%%%%%%%%%%

\newcommand\BitBoxArrowTailInset{-.9}
\newcommand\BitBoxArrowHeadInset{-16.7-\BitBoxArrowTailInset}




%%%%%%%%%%%%%%%%%%%%%%%%%%%%%%%%%%%%%%%%%%%%%%%%%%%%%%%%%%%%%
%%%%%%%%%%%%%%%%%%%%%%%%%%%%%%%%%%%%%%%%%%%%%%%%%%%%%%%%%%%%%
%%%%%%%%%%%%%%%%%%%%%%%%%%%%%%%%%%%%%%%%%%%%%%%%%%%%%%%%%%%%%
%%%%%%%%%%%%%%%%%%%%%%%%%%%%%%%%%%%%%%%%%%%%%%%%%%%%%%%%%%%%%
% Draw hex markers with a baseline at zero
% #1 The number of bits in the box
\newcommand\TheHexMark[1]{
	\draw [line width=.5mm] (#1+.5,0) -- (#1+.5, .4);
}

\newcommand\DrawHexMarkersRel[1]{
	\pgfmathsetmacro\num{int(#1)}
	\foreach \x in {0,4,...,\num}
		\draw [line width=.5mm] (\x+.5,0) -- (\x+.5, .4);
}

%%%%%%%%%%%%%%%%%%%%%%%%%%%%%%%%%%%%%%%%%%%%%%%%%%%%%%%%%%%%%
% Draw an instruction box with a baseline at zero
% #1 MSB position
% #2 LSB position
% #3 the segment label
\newcommand\DrawInsnBoxRelTop[3]{
	\pgfmathsetmacro\leftpos{int(32-#1)}
	\pgfmathsetmacro\rightpos{int(32-#2)}
	\draw (\leftpos-.5,1.5) -- (\rightpos+.5,1.5);	% box top
	\draw (\leftpos-.5,0) -- (\leftpos-.5, 1.5);	% left end
	\draw (\rightpos+.5,0) -- (\rightpos+.5, 1.5);	% right end
	\pgfmathsetmacro\posn{32-#1+(#1-#2)/2}
	\node at (\posn,.75) {\small#3};				% the field label
}

% Draw only the bottom line of an instruction box with a baseline at zero
% #1 MSB position
% #2 LSB position
\newcommand\DrawInsnBoxRelBottom[2]{
	\pgfmathsetmacro\leftpos{int(32-#1)}
	\pgfmathsetmacro\rightpos{int(32-#2)}
	\draw (\leftpos-.5,0) -- (\rightpos+.5,0);		% box bottom
}

%%%%%%%%%%%%%%%%%%%%%%%%%%%%%%%%%%%%%%%%%%%%%%%%%%%%%%%%%%%%%
% Draw an instruction box with a baseline at zero
% #1 MSB position
% #2 LSB position
% #3 the segment label
\newcommand\DrawInsnBoxRel[3]{
	\DrawInsnBoxRelTop{#1}{#2}{#3}
	\DrawInsnBoxRelBottom{#1}{#2}
}

% #1 MSB position
% #2 LSB position
\newcommand\DrawInsnBoxCastle[2]{
	\pgfmathsetmacro\leftpos{int(32-#1)}
	\pgfmathsetmacro\rightpos{int(32-#2)}
	\draw (\leftpos-.5,0) -- (\leftpos-.5, .75);		% left end
	\draw (\rightpos+.5,0) -- (\rightpos+.5, .75);	% right end
	\node at (\leftpos,.5) {\tiny#1};
	\ifthenelse{\equal{#1}{#2}}
	{}
	{ \draw(\rightpos,.5) node{\tiny#2}; }
}


%%%%%%%%%%%%%%%%%%%%%%%%%%%%%%%%%%%%%%%%%%%%%%%%%%%%%%%%%%%%%
\newcommand\DrawDMTrack[1]{
    \StrLen{#1}[\numchars]
    \begin{scope}[shift={(0,.75)}]
    	\DrawInsnBitstring{\numchars}{#1}{CP/M Track}
		\DrawInsnBoxSeg{\numchars}{15}{0}{track}
    \end{scope}

    \DrawHexMarkersRel{\numchars}
}


\newcommand\DrawInsnBoxLabelsDMTag{
	\DrawInsnBoxSeg{\numchars}{7}{7}{inv}
	\DrawInsnBoxSeg{\numchars}{6}{6}{}
	\DrawInsnBoxSeg{\numchars}{5}{0}{track hi}
}
\newcommand\DrawDMTag[1]{
    \StrLen{#1}[\numchars]
    \begin{scope}[shift={(0,.75)}]
    	\DrawInsnBitstring{\numchars}{#1}{Tag}
		\DrawInsnBoxLabelsDMTag
    \end{scope}
    \DrawHexMarkersRel{\numchars}
}




\newcommand\DrawDMSlot[1]{
    \StrLen{#1}[\numchars]
    \begin{scope}[shift={(0,.75)}]
    	\DrawInsnBitstring{\numchars}{#1}{DM Slot}
		\DrawInsnBoxSeg{\numchars}{15}{0}{slot}
    \end{scope}
    \DrawHexMarkersRel{\numchars}
}

\newcommand\DrawDMSlotAddress[1]{
    \StrLen{#1}[\numchars]
    \begin{scope}[shift={(0,.75)}]
    	\DrawInsnBitstring{\numchars}{#1}{RAM Address}
		\DrawInsnBoxSeg{\numchars}{15}{0}{slot\_address}
    \end{scope}
    \DrawHexMarkersRel{\numchars}
}

\newcommand\DrawDMRamBank[1]{
    \StrLen{#1}[\numchars]
    \begin{scope}[shift={(0,.75)}]
    	\DrawInsnBitstring{\numchars}{#1}{RAM Bank}
		\DrawInsnBoxSeg{\numchars}{7}{4}{bank\_addr}
		\DrawInsnBoxSeg{\numchars}{3}{0}{}
    \end{scope}
    \DrawHexMarkersRel{\numchars}
}

% #1 total characters/width
% #2 MSB position
% #3 LSB position
% #4 the segment label
\newcommand\DrawInsnBoxSegTop[4]{
    \pgfmathsetmacro\leftpos{int(#1-#2)}
    \pgfmathsetmacro\rightpos{int(#1-#3)}

%    \draw (\leftpos-.5,-.75) -- (\rightpos+.5,-.75);    % box bottom
    \draw (\leftpos-.5,1.75) -- (\rightpos+.5,1.75);    % box top
    \draw (\leftpos-.5,.25) -- (\leftpos-.5, 2.5);     % left end
    \draw (\rightpos+.5,.25) -- (\rightpos+.5, 2.5);   % right end
    \node at (\leftpos,2.2) {\tiny#2};
    \draw(\rightpos,2.2) node{\tiny#3};

    \pgfmathsetmacro\posn{#1-#2+(#2-#3)/2}
    \pgfmathsetmacro\range{int(#2-#3+1)}
    \node at (\posn,1.2) {\small#4};            % the field label

%    \begin{scope}[shift={(0,-.7)}]\InsnBoxFieldWidthArrow{#2}{#3}\end{scope}
}

% #1 total characters/width
% #2 MSB position
% #3 LSB position
\newcommand\DrawInsnBoxSegBtm[3]{
    \pgfmathsetmacro\leftpos{int(#1-#2)}
    \pgfmathsetmacro\rightpos{int(#1-#3)}

    \draw (\leftpos-.5,-.75) -- (\rightpos+.5,-.75);    % box bottom
    \draw (\leftpos-.5,-.75) -- (\leftpos-.5, 2.5);     % left end
    \draw (\rightpos+.5,-.75) -- (\rightpos+.5, 2.5);   % right end

	\begin{scope}[shift={(0,-.7)}]\InsnBoxFieldWidthArrow{#1}{#2}{#3}\end{scope}
}

% #1 total characters/width
% #2 MSB position
% #3 LSB position
\newcommand\DrawInsnBoxSegMid[3]{
    \pgfmathsetmacro\leftpos{int(#1-#2)}
    \pgfmathsetmacro\rightpos{int(#1-#3)}

    \draw (\leftpos-.5,-.5) -- (\leftpos-.5, .75);     % left end
    \draw (\rightpos+.5,-.5) -- (\rightpos+.5, .75);   % right end
}


\newcommand\DrawTagTop{
	\DrawInsnBoxSegTop{8}{7}{7}{inv}
	\DrawInsnBoxSegTop{8}{6}{6}{}
	\DrawInsnBoxSegTop{8}{5}{0}{tag}
}

% #1 bit string value
\newcommand\DrawTagMid[1]{
	\StrLen{#1}[\numchars]
	\DrawInsnBoxSegMid{\numchars}{7}{7}
	\DrawInsnBoxSegMid{\numchars}{6}{6}
	\DrawInsnBoxSegMid{\numchars}{5}{0}

	\DrawInsnBitstring{\numchars}{#1}{}
}

\newcommand\DrawTagBtm{
	\DrawInsnBoxSegBtm{8}{7}{7}
	\DrawInsnBoxSegBtm{8}{6}{6}
	\DrawInsnBoxSegBtm{8}{5}{0}
    \begin{scope}[shift={(0,-.75)}]\DrawHexMarkersRel{\numchars}\end{scope}
}


\newcommand\DrawTagTable{
	\begin{scope}[shift={(0,.75)}]
		\StrLen{v0tttttt}[\numchars]
		\DrawTagTop

		\draw [decorate, decoration = {calligraphic brace}] (0,-7) -- (0,.5);
		\node[text width=1cm, align=left] at (-.25, -3.25) {\small 256};

		\node[text width=3cm, align=right] at (9, -5) {Tag Table};


		\DrawTagMid{v0cdefgh}
		\begin{scope}[shift={(0,-1.25)}]\DrawTagMid{v0cdefgh}\end{scope}
		\begin{scope}[shift={(0,-2.5)}]\DrawTagMid{v0cdefgh}\end{scope}
		\begin{scope}[shift={(0,-3.75)}]\DrawTagMid{v0cdefgh}\end{scope}
		\begin{scope}[shift={(0,-5.0)}]\DrawTagMid{. . . . }\end{scope}
		\begin{scope}[shift={(0,-6.25)}]
			\DrawTagMid{v0cdefgh}
			\DrawTagBtm
		\end{scope}

	\end{scope}
}


\newcommand\DrawTrackDecode{
	\BeginTikzPicture
	\begin{scope}[shift={(0,-1.5)}]

		\DrawDMTrack{abcdefghijklmnpq}

		\pgfmathsetmacro\ArrowNorth{\BitBoxArrowTailInset}
		\pgfmathsetmacro\ArrowSouth{\BitBoxArrowHeadInset}

		% arrows from track to tag 
		\draw[blue,->](3,\ArrowNorth)to[out=270,in=90](3,\ArrowSouth);
		\draw[blue,->](4,\ArrowNorth)to[out=270,in=90](4,\ArrowSouth);
		\draw[blue,->](5,\ArrowNorth)to[out=270,in=90](5,\ArrowSouth);
		\draw[blue,->](6,\ArrowNorth)to[out=270,in=90](6,\ArrowSouth);
		\draw[blue,->](7,\ArrowNorth)to[out=270,in=90](7,\ArrowSouth);
		\draw[blue,->](8,\ArrowNorth)to[out=270,in=90](8,\ArrowSouth);


		% arrows from track to slot 
		\draw[blue,->](9,\ArrowNorth)to[out=270,in=90](21,\ArrowSouth);
		\draw[blue,->](10,\ArrowNorth)to[out=270,in=90](22,\ArrowSouth);
		\draw[blue,->](11,\ArrowNorth)to[out=270,in=90](23,\ArrowSouth);
		\draw[blue,->](12,\ArrowNorth)to[out=270,in=90](24,\ArrowSouth);
		\draw[blue,->](13,\ArrowNorth)to[out=270,in=90](25,\ArrowSouth);
		\draw[blue,->](14,\ArrowNorth)to[out=270,in=90](26,\ArrowSouth);
		\draw[blue,->](15,\ArrowNorth)to[out=270,in=90](27,\ArrowSouth);
		\draw[blue,->](16,\ArrowNorth)to[out=270,in=90](28,\ArrowSouth);

	\end{scope}


	% tag table
	\begin{scope}[shift={(0,-21)}]
		\DrawTagTable
	\end{scope}

	% slot number
	\begin{scope}[shift={(12,-21)}]
		\DrawDMSlot{00000000ijklmnpq}

		\pgfmathsetmacro\ArrowNorth{\BitBoxArrowTailInset}
		\pgfmathsetmacro\ArrowSouth{\BitBoxArrowHeadInset}

		% arrows into the RAM bank
		\draw[red,->](9,\ArrowNorth)to[out=270,in=90](-11,-14);
		\draw[red,->](10,\ArrowNorth)to[out=270,in=90](-10,-14);

		% arrows into the RAM address
		\draw[blue,->](11,\ArrowNorth)to[out=270,in=90](2,-14);
		\draw[blue,->](12,\ArrowNorth)to[out=270,in=90](3,-14);
		\draw[blue,->](13,\ArrowNorth)to[out=270,in=90](4,-14);
		\draw[blue,->](14,\ArrowNorth)to[out=270,in=90](5,-14);
		\draw[blue,->](15,\ArrowNorth)to[out=270,in=90](6,-14);
		\draw[blue,->](16,\ArrowNorth)to[out=270,in=90](7,-14);

		% the arrow into the tag table
		\draw[blue,->](.5,.75)to[out=180,in=0](-3.5,-3);

	\end{scope}

	% bank number
	\begin{scope}[shift={(-2,-38.5)}]
		\DrawDMRamBank{00ij0000}
	\end{scope}

	% slot RAM address
	\begin{scope}[shift={(12,-38.5)}]
		\DrawDMSlotAddress{0klmnpq000000000}
	\end{scope}

	\EndTikzPicture
}






\clearpage

\begin{center}
{\huge Z80-Retro! Direct Mapped Cache}
\end{center}
\vspace{.5in}

\begin{center}
\DrawTrackDecode
\end{center}







\end{document}
